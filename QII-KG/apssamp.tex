% !TeX root = apssamp.tex
% ****** Start of file apssamp.tex ******
%
%   This file is part of the APS files in the REVTeX 4.2 distribution.
%   Version 4.2a of REVTeX, December 2014
%
%   Copyright (c) 2014 The American Physical Society.
%
%   See the REVTeX 4 README file for restrictions and more information.
%
% TeX'ing this file requires that you have AMS-LaTeX 2.0 installed
% as well as the rest of the prerequisites for REVTeX 4.2
%
% See the REVTeX 4 README file
% It also requires running BibTeX. The commands are as follows:
%
%  1)  latex apssamp.tex
%  2)  bibtex apssamp
%  3)  latex apssamp.tex
%  4)  latex apssamp.tex
%
\documentclass[a4paper,14pt]{extarticle}
\linespread{1.5}
\usepackage{graphicx}% Include figure files
\usepackage{dcolumn}% Align table columns on decimal point
\usepackage{bm}% bold math
\usepackage{hyperref}% add hypertext capabilities
\usepackage[mathlines]{lineno}% Enable numbering of text and display math
%\linenumbers\relax % Commence numbering lines
\usepackage{amsmath}
\usepackage{amssymb}

%\usepackage[showframe,%Uncomment any one of the following lines to test 
%%scale=0.7, marginratio={1:1, 2:3}, ignoreall,% default settings
%%text={7in,10in},centering,
%%margin=1.5in,
%%total={6.5in,8.75in}, top=1.2in, left=0.9in, includefoot,
%%height=10in,a5paper,hmargin={3cm,0.8in},
%]{geometry}
\usepackage[left=3cm, top=3cm, right=2cm, bottom=2cm]{geometry}
\usepackage{cite}
\usepackage[brazilian, portuguese]{babel}

\title{Construção história da equação de Klein-Gordon\\ Mecânica quantica II}% Force line breaks with \\

\author{Tiago Henrique Barbosa Alves}
\date{\today}


\begin{document}

\maketitle


\section{Introdução}

\paragraph{} A equação de Klein-Gordon foi primeiramente considerada como uma equação de uma
onda quantica por Schrödinger, ao 
tratar uma equação para descrever as ondas de de Broglie, a equação surge em suas notações em 1925 em uma tentativa 
de aplicar ao átomo de hidrogênio. Porém a equação falho ao não levar em consideração o spin do elétron,
falhando em predizer a estrutura fina do átomo de hidrogênio. Posteriormente Schrödinger publicou apenas uma versão não 
relatívistica da equação que predizia os os niveis de energia do átomo de Bohr sem a estrutura fina.\cite{Kragh1984}

Em 1926 a equação de Klein-Gordon surge simultâneamente nos trabalhos de Klein\cite{Klein:1926fj},
Gordon\cite{Gordon1926}, Fock, de Donder e van den Dungen\cite{deDonder1926}. Sendo a junção dos trabalhos de Klein e Gordon tornado-se mais
popular entre os físicos, enquanto que o trabalho de de Donede e van den Dungen não por não ter introduzido 
a teoria quantica explicitamente e não fazia nenhuma referências de trabalhos física quântica acabou por receber menor destaque.


\section{Desenvolvimento matemático}

\paragraph{} A equação de Klein-Gordon pode ser obtida através da relação energia-momento
\begin{equation}
  E=\frac{p^2}{2m} 
\end{equation}
onde pelo princípio da correspondência $p\rightarrow -i \hbar \nabla $ e $E \rightarrow i \hbar \frac{\partial}{\partial t}$, onde 
obtemos a equação de Schrödingerpara uma partícula livre.

\begin{equation}
  i\hbar \frac{\partial \Psi}{\partial t} = -\frac{\hbar ^2}{2m} \nabla^2 \Psi .
\end{equation}

Repetindo o processo para o análogo relatívistico da equação de energia momento $E^2 = p^2 + \left(\frac{mc}{\hbar}\right)^2$. Aplicando o princípio da correspondência 
obtemos a equação de Klein-Gordon

\begin{equation}
  \left[\square + \left(\frac{mc}{\hbar}\right)\right]\Psi = 0, 
\end{equation}
onde $\square = \partial_\mu \partial^{\mu}$ é o operador d'Alambertiano. Para uma dedução detalhada, a partir de princípios mínimos é possivel consultar \cite[Groessing]{groessing2002derivation}.

% The \nocite command causes all entries in a bibliography to be printed out
% whether or not they are actually referenced in the text. This is appropriate
% for the sample file to show the different styles of references, but authors
% most likely will not want to use it.
\bibliography{apssamp}{}
\bibliographystyle{plain}

\end{document}
%
% ****** End of file apssamp.tex ******
