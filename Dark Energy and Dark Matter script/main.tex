% !TeX root = main.tex

% abtex2-modelo-artigo.tex, v-1.9.2 laurocesar
% Copyright 2012-2014 by abnTeX2 group at http://abntex2.googlecode.com/ 
%

% ------------------------------------------------------------------------
% ------------------------------------------------------------------------
% abnTeX2: Modelo de Artigo Acadêmico em conformidade com
% ABNT NBR 6022:2003: Informação e documentação - Artigo em publicação 
% periódica científica impressa - Apresentação
% ------------------------------------------------------------------------
% ------------------------------------------------------------------------

\documentclass[
	% -- opções da classe memoir --
	article,			% indica que é um artigo acadêmico
	11pt,				% tamanho da fonte
	oneside,			% para impressão apenas no verso. Oposto a twoside
	a4paper,			% tamanho do papel. 
	% -- opções da classe abntex2 --
	%chapter=TITLE,		% títulos de capítulos convertidos em letras maiúsculas
	%section=TITLE,		% títulos de seções convertidos em letras maiúsculas
	%subsection=TITLE,	% títulos de subseções convertidos em letras maiúsculas
	%subsubsection=TITLE % títulos de subsubseções convertidos em letras maiúsculas
	% -- opções do pacote babel --
	english,			% idioma adicional para hifenização
	brazil,				% o último idioma é o principal do documento
	sumario=tradicional
	]{abntex2}


% ---
% PACOTES
% ---

% ---
% Pacotes fundamentais 
% ---
\usepackage{lmodern}			% Usa a fonte Latin Modern
\usepackage[T1]{fontenc}		% Selecao de codigos de fonte.
\usepackage[utf8]{inputenc}		% Codificacao do documento (conversão automática dos acentos)
\usepackage{indentfirst}		% Indenta o primeiro parágrafo de cada seção.
\usepackage{nomencl} 			% Lista de simbolos
\usepackage{color}				% Controle das cores
\usepackage{graphicx}			% Inclusão de gráficos
\usepackage{microtype} 			% para melhorias de justificação
% ---
		
% ---
% Pacotes adicionais, usados apenas no âmbito do Modelo Canônico do abnteX2
% ---
\usepackage{lipsum}				% para geração de dummy text
% ---
		
% ---
% Pacotes de citações
% ---
\usepackage[brazilian,hyperpageref]{backref}	 % Paginas com as citações na bibl
\usepackage[alf]{abntex2cite}	% Citações padrão ABNT
% ---
% ---
% Configurações do pacote backref
% Usado sem a opção hyperpageref de backref
\renewcommand{\backrefpagesname}{Citado na(s) página(s):~}
% Texto padrão antes do número das páginas
\renewcommand{\backref}{}
% Define os textos da citação
\renewcommand*{\backrefalt}[4]{
	\ifcase #1 %
		Nenhuma citação no texto.%
	\or
		Citado na página #2.%
	\else
		Citado #1 vezes nas páginas #2.%
	\fi}%
% ---

% ---
% Informações de dados para CAPA e FOLHA DE ROSTO
% ---
\titulo{Roteiro:\\ Energia Escura}
\autor{Tiago H. B. Alves\texorpdfstring{\thanks{tiagohbalves@hotmail.com}}{}}
\local{Brasil}
\data{2024}
\def\assunto{Assunto}
% ---

% ---
% Configurações de aparência do PDF final

% alterando o aspecto da cor azul
\definecolor{blue}{RGB}{41,5,195}

% informações do PDF
\makeatletter
\hypersetup{
     	%pagebackref=true,
		pdftitle={\@title}, 
		pdfauthor={\@author},
    	pdfsubject={\assunto},
	    pdfcreator={\@author},
		pdfkeywords={abnt}{latex}{abntex}{abntex2}{atigo científico}, 
		colorlinks=true,       		% false: boxed links; true: colored links
    	linkcolor=blue,          	% color of internal links
    	citecolor=blue,        		% color of links to bibliography
    	filecolor=magenta,      		% color of file links
		urlcolor=blue,
		bookmarksdepth=4
}
\makeatother
% --- 


\ifthenelse{\equal{\ABNTEXisarticle}{true}}{%
\renewcommand{\maketitlehookb}{}
}{}
% ---
% compila o indice
% ---
\makeindex
% ---

% ---
% Altera as margens padrões
% ---
\setlrmarginsandblock{3cm}{3cm}{*}
\setulmarginsandblock{3cm}{3cm}{*}
\checkandfixthelayout
% ---

% --- 
% Espaçamentos entre linhas e parágrafos 
% --- 

% O tamanho do parágrafo é dado por:
\setlength{\parindent}{1.3cm}

% Controle do espaçamento entre um parágrafo e outro:
\setlength{\parskip}{0.2cm}  % tente também \onelineskip

% Espaçamento simples
\SingleSpacing

% ----
% Início do documento
% ----
\begin{document}

% Retira espaço extra obsoleto entre as frases.
\frenchspacing 

% ----------------------------------------------------------
% ELEMENTOS PRÉ-TEXTUAIS
% ----------------------------------------------------------

%---
%
% Se desejar escrever o artigo em duas colunas, descomente a linha abaixo
% e a linha com o texto ``FIM DE ARTIGO EM DUAS COLUNAS''.
% \twocolumn[    		% INICIO DE ARTIGO EM DUAS COLUNAS
%
%---
% página de titulo
\maketitle

% resumo em português
\begin{resumoumacoluna}
 Conforme a ABNT NBR 6022:2003, o resumo é elemento obrigatório, constituído de
 uma sequência de frases concisas e objetivas e não de uma simples enumeração
 de tópicos, não ultrapassando 250 palavras, seguido, logo abaixo, das palavras
 representativas do conteúdo do trabalho, isto é, palavras-chave e/ou
 descritores, conforme a NBR 6028. (\ldots) As palavras-chave devem figurar logo
 abaixo do resumo, antecedidas da expressão Palavras-chave:, separadas entre si por
 ponto e finalizadas também por ponto.
 
 \vspace{\onelineskip}
 
 \noindent
 \textbf{Palavras-chaves}: latex. abntex. editoração de texto.
\end{resumoumacoluna}

% ]  				% FIM DE ARTIGO EM DUAS COLUNAS
% ---

% ----------------------------------------------------------
% ELEMENTOS TEXTUAIS
% ----------------------------------------------------------
\textual

% ----------------------------------------------------------
% Introdução
% ----------------------------------------------------------
\section*{Introdução}
\addcontentsline{toc}{section}{Introdução}
O universo é composto de 4 forças fundamentais, que são elas, fraca, forte, eletromagnética e gravitacional. Cada qual 
corresponde à um papel fundamental no universo. 

A força forte é responsável 


Ao contemplarmos o universo, nos deparamos com imagens magníficas de astros, estrelas, nebulosas e galáxias. No entanto, essa parte visível do universo corresponde a menos de 5\% do todo. A maior parte do universo é composta por uma componente misteriosa que chamamos de energia escura.

A energia escura é indetectável para nós e seus efeitos na escala local são imperceptíveis. Apesar disso, ela é a grande responsável pela expansão acelerada do universo, tendo um papel crucial na sua evolução como um todo. Os cientistas ainda não sabem exatamente do que a energia escura é composta ou como ela funciona, mas existem diversas teorias que tentam explicar sua natureza.

A energia escura é um dos maiores mistérios da ciência moderna. Desvendar seus segredos é um dos grandes desafios dos cientistas do século XXI. Qual será o destino do universo se a energia escura continuar a se expandir? Será que um dia conseguiremos desvendar os mistérios da energia escura?
Motivação, origem, por que?
\section{Origem historica}
Quando Einstein apresentou a sua equação para a relatividade geral, um conjunto total de 16 equações sumarizadas por:
\begin{equation}
	R_{\mu \nu}+\frac{1}{2}g_{\mu \nu} R = T_{\mu \nu}
\end{equation}
\section{Energia escura}

\section{Evidencias}
\section{Candidatos}
\end{document}