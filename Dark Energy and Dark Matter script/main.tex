% !TeX root = main.tex

% abtex2-modelo-artigo.tex, v-1.9.2 laurocesar
% Copyright 2012-2014 by abnTeX2 group at http://abntex2.googlecode.com/ 
%

% ------------------------------------------------------------------------
% ------------------------------------------------------------------------
% abnTeX2: Modelo de Artigo Acadêmico em conformidade com
% ABNT NBR 6022:2003: Informação e documentação - Artigo em publicação 
% periódica científica impressa - Apresentação
% ------------------------------------------------------------------------
% ------------------------------------------------------------------------

\documentclass[
	% -- opções da classe memoir --
	article,			% indica que é um artigo acadêmico
	11pt,				% tamanho da fonte
	oneside,			% para impressão apenas no verso. Oposto a twoside
	a4paper,			% tamanho do papel. 
	% -- opções da classe abntex2 --
	%chapter=TITLE,		% títulos de capítulos convertidos em letras maiúsculas
	%section=TITLE,		% títulos de seções convertidos em letras maiúsculas
	%subsection=TITLE,	% títulos de subseções convertidos em letras maiúsculas
	%subsubsection=TITLE % títulos de subsubseções convertidos em letras maiúsculas
	% -- opções do pacote babel --
	english,			% idioma adicional para hifenização
	brazil,				% o último idioma é o principal do documento
	sumario=tradicional
	]{abntex2}


% ---
% PACOTES
% ---

% ---
% Pacotes fundamentais 
% ---
\usepackage{lmodern}			% Usa a fonte Latin Modern
\usepackage[T1]{fontenc}		% Selecao de codigos de fonte.
\usepackage[utf8]{inputenc}		% Codificacao do documento (conversão automática dos acentos)
\usepackage{indentfirst}		% Indenta o primeiro parágrafo de cada seção.
\usepackage{nomencl} 			% Lista de simbolos
\usepackage{color}				% Controle das cores
\usepackage{graphicx}			% Inclusão de gráficos
\usepackage{microtype} 			% para melhorias de justificação
% ---
		
% ---
% Pacotes adicionais, usados apenas no âmbito do Modelo Canônico do abnteX2
% ---
\usepackage{lipsum}				% para geração de dummy text
% ---
		
% ---
% Pacotes de citações
% ---
\usepackage[brazilian,hyperpageref]{backref}	 % Paginas com as citações na bibl
\usepackage[alf]{abntex2cite}	% Citações padrão ABNT
% ---
% ---
% Configurações do pacote backref
% Usado sem a opção hyperpageref de backref
\renewcommand{\backrefpagesname}{Citado na(s) página(s):~}
% Texto padrão antes do número das páginas
\renewcommand{\backref}{}
% Define os textos da citação
\renewcommand*{\backrefalt}[4]{
	\ifcase #1 %
		Nenhuma citação no texto.%
	\or
		Citado na página #2.%
	\else
		Citado #1 vezes nas páginas #2.%
	\fi}%
% ---

% ---
% Informações de dados para CAPA e FOLHA DE ROSTO
% ---
\titulo{Roteiro:\\ Energia Escura}
\autor{Tiago H. B. Alves\texorpdfstring{\thanks{tiagohbalves@hotmail.com}}{}}
\local{Brasil}
\data{2024}
\def\assunto{Assunto}
% ---

% ---
% Configurações de aparência do PDF final

% alterando o aspecto da cor azul
\definecolor{blue}{RGB}{41,5,195}

% informações do PDF
\makeatletter
\hypersetup{
     	%pagebackref=true,
		pdftitle={\@title}, 
		pdfauthor={\@author},
    	pdfsubject={\assunto},
	    pdfcreator={\@author},
		pdfkeywords={abnt}{latex}{abntex}{abntex2}{atigo científico}, 
		colorlinks=true,       		% false: boxed links; true: colored links
    	linkcolor=blue,          	% color of internal links
    	citecolor=blue,        		% color of links to bibliography
    	filecolor=magenta,      		% color of file links
		urlcolor=blue,
		bookmarksdepth=4
}
\makeatother
% --- 


\ifthenelse{\equal{\ABNTEXisarticle}{true}}{%
\renewcommand{\maketitlehookb}{}
}{}
% ---
% compila o indice
% ---
\makeindex
% ---

% ---
% Altera as margens padrões
% ---
\setlrmarginsandblock{3cm}{3cm}{*}
\setulmarginsandblock{3cm}{3cm}{*}
\checkandfixthelayout
% ---

% --- 
% Espaçamentos entre linhas e parágrafos 
% --- 

% O tamanho do parágrafo é dado por:
\setlength{\parindent}{1.3cm}

% Controle do espaçamento entre um parágrafo e outro:
\setlength{\parskip}{0.2cm}  % tente também \onelineskip

% Espaçamento simples
\SingleSpacing

% ----
% Início do documento
% ----
\begin{document}

% Retira espaço extra obsoleto entre as frases.
\frenchspacing 

% ----------------------------------------------------------
% ELEMENTOS PRÉ-TEXTUAIS
% ----------------------------------------------------------

%---
%
% Se desejar escrever o artigo em duas colunas, descomente a linha abaixo
% e a linha com o texto ``FIM DE ARTIGO EM DUAS COLUNAS''.
% \twocolumn[    		% INICIO DE ARTIGO EM DUAS COLUNAS
%
%---
% página de titulo
\maketitle

% resumo em português
\begin{resumoumacoluna}
 Conforme a ABNT NBR 6022:2003, o resumo é elemento obrigatório, constituído de
 uma sequência de frases concisas e objetivas e não de uma simples enumeração
 de tópicos, não ultrapassando 250 palavras, seguido, logo abaixo, das palavras
 representativas do conteúdo do trabalho, isto é, palavras-chave e/ou
 descritores, conforme a NBR 6028. (\ldots) As palavras-chave devem figurar logo
 abaixo do resumo, antecedidas da expressão Palavras-chave:, separadas entre si por
 ponto e finalizadas também por ponto.
 
 \vspace{\onelineskip}
 
 \noindent
 \textbf{Palavras-chaves}: latex. abntex. editoração de texto.
\end{resumoumacoluna}

% ]  				% FIM DE ARTIGO EM DUAS COLUNAS
% ---

% ----------------------------------------------------------
% ELEMENTOS TEXTUAIS
% ----------------------------------------------------------
\textual

% ----------------------------------------------------------
% Introdução
% ----------------------------------------------------------
\section*{Introdução}
\addcontentsline{toc}{section}{Introdução}
O universo é composto de 4 forças fundamentais, que são elas, fraca, forte, eletromagnética e gravitacional. Cada qual 
corresponde à um papel fundamental no universo. 

A força forte é responsável 


Ao contemplarmos o universo, nos deparamos com imagens magníficas de astros, estrelas, nebulosas e galáxias. No entanto, essa parte visível do universo corresponde a menos de 5\% do todo. A maior parte do universo é composta por uma componente misteriosa que chamamos de energia escura.

A energia escura é indetectável para nós e seus efeitos na escala local são imperceptíveis. Apesar disso, ela é a grande responsável pela expansão acelerada do universo, tendo um papel crucial na sua evolução como um todo. Os cientistas ainda não sabem exatamente do que a energia escura é composta ou como ela funciona, mas existem diversas teorias que tentam explicar sua natureza.

A energia escura é um dos maiores mistérios da ciência moderna. Desvendar seus segredos é um dos grandes desafios dos cientistas do século XXI. Qual será o destino do universo se a energia escura continuar a se expandir? Será que um dia conseguiremos desvendar os mistérios da energia escura?
Motivação, origem, por que?
\section{Origem historica}
Quando Einstein apresentou a sua equação para a relatividade geral, um conjunto total de 16 equações sumarizadas por:
\begin{equation}
	R_{\mu \nu}+\frac{1}{2}g_{\mu \nu} R = T_{\mu \nu}
\end{equation}
\section{Energia escura}

\section{Evidencias}
\section{Candidatos}

\section{Matéria escura}

A matéria escura é um dos principais problemas em aberto da física moderna e da cosmologia, a primeira indicação da sua presença
pode ser encontrada no estudo da dinâmica de nossa galáxia. Em 1922 Jacobus Kaptey calculo a densidade
de matéria perto do Sol e também estimou a densidade de todo o plano galático\cite{1922ApJ55302K}. Kaptey concluiu que a densidade
de matéria obtida em seus cálculos era o suficiente para explicar o movimento vertical das estrelas pŕoximas ao plano galáctico. Porém no mesmo ano, James Jeans
ao reanalizar o movimento dessas estrelas no plano vertical, concluiu que deveria haver duas "estrelas escuras" para cada estrela\cite{10.1093/mnras/82.3.122}, ou em outras palavras a massa total 
da galáxia deveria ser o triplo do observado.

Em 1933 Fritz Zwicky ao medir as velocidades radiais no Superaglomerado de Coma, na constelação de Coma Berenice. Ao calcular a velocidade orbital das galáxias
Zwicky observou que a massa do cluster deveria ser maior, por um fator de 10, da massa observada do cluster\cite{1933AcHPh...6..110Z}. Ele concluiu que deveria existir uma 
"matéria escura", invisível para nós, para que o aglomerado tivesse força gravitacional suficiente para manter as galáxias em órbita. No ano anterior o astrônomo 
Jan Oort chegou á resultados semelhanets ao de Zwicky porém ele não postulou a existência de nenhuma matéra exótica como a matéria escura, porém outras evidências 
da matéria escura refutam as ideias propostas por Oort.

A ideia foi ressuscitada apenas nos anos 1970,
quando astrônomos observaram o movimento de galáxias-satélite — pequenas
galáxias próximas às grandes —, que só podiam ser explicadas pela presença de
uma
matéria
adicional
e
invisível.
Essas
e
outras
observações
começaram
a
transformar a matéria escura em tópico de pesquisa séria.
Mas seu status se concretizou de verdade a partir dos trabalhos de Vera Rubin,
astrônoma
do
Instituto
Carnegie,
de
Washington,

,
que
trabalhou
com
o
astrônomo Kent Ford. Depois da pós-graduação na Universidade de Georgetown,
Rubin decidiu medir o movimento angular de estrelas em galáxias — a começar
por
Andrômeda
—,
em
parte
para
não
pisar
nos
territórios
fechados
e
superprotegidos de outros cientistas. Ela mudou de rumo na pesquisa depois de
sua
tese
—
que
media
velocidades
de
galáxias
e
confirmou
a
existência
de
aglomerados — ter sido rejeitada pela maior parte da comunidade científica, um
pouco pelo motivo deselegante de invadir os domínios científicos de outros. Em
seu trabalho de pós-graduação, Rubin decidiu entrar em um campo de pesquisa
menos concorrido, e assim resolveu estudar a velocidade orbital de estrelas.
A decisão de Rubin levou à descoberta que talvez seja a mais emocionante de
sua época. Nos anos 1970, ela e Kent Ford, seu colaborador, descobriram que as
velocidades rotacionais das estrelas eram quase as mesmas a qualquer distância
do centro galáctico. Ou seja, as estrelas rotacionavam em velocidade constante,
mesmo que muito distantes da região que continha matéria luminosa. A única
explicação possível era uma matéria ainda não reconhecida que ajudava a refrear
as estrelas mais distantes e se movimentava bem mais rápido que o esperado. Sem
essa
contribuição
adicional,
as
estrelas
com
as
velocidades
que
Rubin
e
Ford
haviam medido sairiam voando em disparada da galáxia. A dedução notável dos
pesquisadores foi que a matéria comum respondia por apenas um sexto da massa
necessária para mantê-la em órbita. As observações de Rubin e Ford resultaram
na
prova
mais
forte
da
existência
da
matéria
escura
na
época,
e
as
curvas
de
rotação das galáxias continuam sendo uma pista importante.
Desde os anos 1970, as provas da matéria escura e a proporção da densidade
energética
bruta
do
universo
que
ela
carrega
se
tornaram
ainda
mais
fortes
e
calculadas de maneira ainda melhor. Os efeitos dinâmicos que nos possibilitam
aprender sobre a matéria escura incluem a rotação das estrelas nas galáxias, como
acabei
de
descrever.
Contudo,
essas
medições
se
aplicavam
apenas
a
galáxiasespiraladas — galáxias que, como nossa Via Láctea, possuem matéria visível em
um
disco
com
braços
espiralados
que
se
projetam
para
fora.
Outra
categoria
importante é a galáxia elíptica, na qual a matéria luminosa possui um formato
mais bulbiforme. Em galáxias elípticas, assim como se dá com as medições de
Zwicky com aglomerados de galáxias, podem-se medir dispersões de velocidade —
o
quanto
as
velocidades
velocidades
variam
são determinadas
entre
pela
as
estrelas
nas
galáxias.
Já
massa dentro da galáxia, elas
que
essas
substituem
a
medição da massa de uma galáxia. As medições em galáxias elípticas também
demonstraram
que
a
matéria
luminosa
é
insuficiente
para
responder
pela
dinâmica medida em suas estrelas. Além de tudo isso, as medições das dinâmicas
de gás interestelar — o gás que não está contido em estrelas — também levavam à
matéria escura. Como essas medições em específico foram feitas dez vezes mais
longe
dos
centros
das
galáxias
do
que
a
extensão
de
matéria
visível,
elas
demonstravam não só que a matéria escura existe, mas que sua gama se estendia
muito além da parte visível de uma galáxia. Medições da temperatura e densidade
do gás, via raios X, confirmaram esse resultado.

\bibliography{cite}

\end{document}